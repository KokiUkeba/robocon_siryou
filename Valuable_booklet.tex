\documentclass[a4paper,11pt]{jsarticle}

% 数式
\usepackage{amsmath,amsfonts}
\usepackage{bm}
% 画像
\usepackage[dvipdfmx]{graphicx}

%ハイパーリンク
\usepackage[dvipdfmx]{hyperref}
\usepackage{pxjahyper} % (u)pLaTeXのときのみかく
\hypersetup{%
 setpagesize=false,%
 bookmarks=true,%
 bookmarksdepth=tocdepth,%
 bookmarksnumbered=true,%
 colorlinks=true,%
 anchorcolor=black,
 linkcolor=black,
 pdftitle={},%
 pdfsubject={},%
 pdfauthor={},%
 pdfkeywords={}}

\title{SolidWorks2021による3次元CAD}
\author{からくりサークル有志}
\date{\today}

\begin{document}
\maketitle
\clearpage
\tableofcontents
\clearpage
\section{はじめに}
この資料は、主にSOLIDWORKS2021を使い慣れた人に向けて書いたものである。そのため、説明は雑なところがあるかもしれないが、この資料を手にする人ならなんとかなるはずである。
\subsection{エンティティのオフセット}
角パイプ、L字のスケッチに使用できる。
\subsubsection{方法}
 \begin{enumerate}
  \item 欲しい材料の外形をスケッチする。
  \item 'エンティティのオフセット'を選択する。
  \item 寸法欄に材料の厚みを入力する。
 \end{enumerate}

\end{document}
